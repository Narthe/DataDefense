\documentclass[a4paper]{report}

\usepackage[french]{babel}
\usepackage[utf8]{inputenc}
\usepackage{graphicx}

\title{Data Defense \\ Game Design Document}
\author{Schmitt Clément \\ Brasseur Julien \\ Vinard Floriant \\ Mathieu Alexandre}

\begin{document}
\maketitle

\tableofcontents

\chapter{Etude de l'existant}
\section{Critères de comparaison}
\section{The Wikileaks Game}
The wikileaks game traite du thème sur lequel nous voulions nous baser au
depart : le scandale de l'espionnage de la NSA qui a intercepte des millions et
des millions d'echanges telephoniques ou autres dans le monde entier. Dans The
wikileaks game on incarne Edward Snowden qui a mis  jour ce scandale et qui
doit collecter des donnees de la CIA pour pouvoir les diriger au monde entier.
Nous avons finalement decide d'elargir notre thème et de parler de la protection
de la vie privée sur internet plutôt que d'un fait d'actualite en particulier.
Le deuxime jeu Data Dealer" traite exactement du sujet choisi pour notre
serious game  une diㄦence prs, et non des moindres, car il s'agit dans ce jeu
d'accumuler le plus de donnees volees  des utilisateurs d'internet pour avoir
un maximum de controle sur la toile et revendre ces donnees, ce n'est bien sur
absolument pas le message que nous voulions faire passer aux joueurs.
Source :
\textit{http://serious.gameclassification.com/FR/games/17888-The-Wikileaks-
Game/index.html}
\section{2025Exmachina}
\section{Data Dealer}
\section{Sex Quest}

\chapter{Cahier des charges}

\chapter{Conception et Réalisation}
\section{Conception}
\section{Réalisation}

\chapter{Evaluation du jeux}

\end{document}
